% Options for packages loaded elsewhere
\PassOptionsToPackage{unicode}{hyperref}
\PassOptionsToPackage{hyphens}{url}
\PassOptionsToPackage{dvipsnames,svgnames,x11names}{xcolor}
%
\documentclass[
  letterpaper,
]{article}

\usepackage{amsmath,amssymb}
\usepackage{iftex}
\ifPDFTeX
  \usepackage[T1]{fontenc}
  \usepackage[utf8]{inputenc}
  \usepackage{textcomp} % provide euro and other symbols
\else % if luatex or xetex
  \usepackage{unicode-math}
  \defaultfontfeatures{Scale=MatchLowercase}
  \defaultfontfeatures[\rmfamily]{Ligatures=TeX,Scale=1}
\fi
\usepackage{lmodern}
\ifPDFTeX\else  
    % xetex/luatex font selection
  \setmainfont[]{Arial}
\fi
% Use upquote if available, for straight quotes in verbatim environments
\IfFileExists{upquote.sty}{\usepackage{upquote}}{}
\IfFileExists{microtype.sty}{% use microtype if available
  \usepackage[]{microtype}
  \UseMicrotypeSet[protrusion]{basicmath} % disable protrusion for tt fonts
}{}
\makeatletter
\@ifundefined{KOMAClassName}{% if non-KOMA class
  \IfFileExists{parskip.sty}{%
    \usepackage{parskip}
  }{% else
    \setlength{\parindent}{0pt}
    \setlength{\parskip}{6pt plus 2pt minus 1pt}}
}{% if KOMA class
  \KOMAoptions{parskip=half}}
\makeatother
\usepackage{xcolor}
\usepackage[margin=1in]{geometry}
\setlength{\emergencystretch}{3em} % prevent overfull lines
\setcounter{secnumdepth}{5}
% Make \paragraph and \subparagraph free-standing
\ifx\paragraph\undefined\else
  \let\oldparagraph\paragraph
  \renewcommand{\paragraph}[1]{\oldparagraph{#1}\mbox{}}
\fi
\ifx\subparagraph\undefined\else
  \let\oldsubparagraph\subparagraph
  \renewcommand{\subparagraph}[1]{\oldsubparagraph{#1}\mbox{}}
\fi

\providecommand{\tightlist}{%
  \setlength{\itemsep}{0pt}\setlength{\parskip}{0pt}}\usepackage{longtable,booktabs,array}
\usepackage{calc} % for calculating minipage widths
% Correct order of tables after \paragraph or \subparagraph
\usepackage{etoolbox}
\makeatletter
\patchcmd\longtable{\par}{\if@noskipsec\mbox{}\fi\par}{}{}
\makeatother
% Allow footnotes in longtable head/foot
\IfFileExists{footnotehyper.sty}{\usepackage{footnotehyper}}{\usepackage{footnote}}
\makesavenoteenv{longtable}
\usepackage{graphicx}
\makeatletter
\def\maxwidth{\ifdim\Gin@nat@width>\linewidth\linewidth\else\Gin@nat@width\fi}
\def\maxheight{\ifdim\Gin@nat@height>\textheight\textheight\else\Gin@nat@height\fi}
\makeatother
% Scale images if necessary, so that they will not overflow the page
% margins by default, and it is still possible to overwrite the defaults
% using explicit options in \includegraphics[width, height, ...]{}
\setkeys{Gin}{width=\maxwidth,height=\maxheight,keepaspectratio}
% Set default figure placement to htbp
\makeatletter
\def\fps@figure{htbp}
\makeatother

\usepackage{booktabs}
\usepackage{longtable}
\usepackage{array}
\usepackage{multirow}
\usepackage{wrapfig}
\usepackage{float}
\usepackage{colortbl}
\usepackage{pdflscape}
\usepackage{tabu}
\usepackage{threeparttable}
\usepackage{threeparttablex}
\usepackage[normalem]{ulem}
\usepackage{makecell}
\usepackage{xcolor}
\definecolor{srebblue}{HTML}{003087}
\definecolor{domblue}{HTML}{307FE2}
\definecolor{domgreen}{HTML}{84BD00}
\definecolor{domlightblue}{HTML}{00AEC7}
\definecolor{accentorange}{HTML}{FFA300}
\usepackage{float}
\floatplacement{table}{H}
\makeatletter
\@ifpackageloaded{caption}{}{\usepackage{caption}}
\AtBeginDocument{%
\ifdefined\contentsname
  \renewcommand*\contentsname{Table of contents}
\else
  \newcommand\contentsname{Table of contents}
\fi
\ifdefined\listfigurename
  \renewcommand*\listfigurename{List of Figures}
\else
  \newcommand\listfigurename{List of Figures}
\fi
\ifdefined\listtablename
  \renewcommand*\listtablename{List of Tables}
\else
  \newcommand\listtablename{List of Tables}
\fi
\ifdefined\figurename
  \renewcommand*\figurename{Figure}
\else
  \newcommand\figurename{Figure}
\fi
\ifdefined\tablename
  \renewcommand*\tablename{Table}
\else
  \newcommand\tablename{Table}
\fi
}
\@ifpackageloaded{float}{}{\usepackage{float}}
\floatstyle{ruled}
\@ifundefined{c@chapter}{\newfloat{codelisting}{h}{lop}}{\newfloat{codelisting}{h}{lop}[chapter]}
\floatname{codelisting}{Listing}
\newcommand*\listoflistings{\listof{codelisting}{List of Listings}}
\makeatother
\makeatletter
\makeatother
\makeatletter
\@ifpackageloaded{caption}{}{\usepackage{caption}}
\@ifpackageloaded{subcaption}{}{\usepackage{subcaption}}
\makeatother

\usepackage{hyphenat}
\usepackage{ifthen}
\usepackage{calc}
\usepackage{calculator}


\usepackage{geometry}

\usepackage{graphicx}
\usepackage{geometry}
\usepackage{afterpage}
\usepackage{tikz}
\usetikzlibrary{calc}
\usetikzlibrary{fadings}
\usepackage[pagecolor=none]{pagecolor}


% Set the titlepage font families







% Set the coverpage font families

\ifLuaTeX
  \usepackage{selnolig}  % disable illegal ligatures
\fi
\usepackage{bookmark}

\IfFileExists{xurl.sty}{\usepackage{xurl}}{} % add URL line breaks if available
\urlstyle{same} % disable monospaced font for URLs
\hypersetup{
  pdftitle={Literacy in SREB States},
  pdfauthor={Keaton Markey},
  colorlinks=true,
  linkcolor={blue},
  filecolor={Maroon},
  citecolor={Blue},
  urlcolor={Blue},
  pdfcreator={LaTeX via pandoc}}

\title{Literacy in SREB States}
\author{Keaton Markey}
\date{March 14, 2024}

\begin{document}
%%%%% begin titlepage extension code


\begin{titlepage}

%%% TITLE PAGE START

% Set up alignment commands
%Page
\newcommand{\titlepagepagealign}{
\ifthenelse{\equal{left}{right}}{\raggedleft}{}
\ifthenelse{\equal{left}{center}}{\centering}{}
\ifthenelse{\equal{left}{left}}{\raggedright}{}
}


\newcommand{\titleandsubtitle}{
% Title and subtitle
{{\LARGE{\bfseries{\nohyphens{Literacy in SREB States}}}}\par
}%
}
\newcommand{\titlepagetitleblock}{
\titleandsubtitle
}

\newcommand{\authorstyle}[1]{{\large{#1}}}

\newcommand{\affiliationstyle}[1]{{\large{#1}}}

\newcommand{\titlepageauthorblock}{
{\authorstyle{\nohyphens{Keaton Markey}\\}}
}

\newcommand{\titlepageaffiliationblock}{
\hangindent=1em
\hangafter=1
\affiliationstyle{{1}.~Southern Regional Education Board,~592 10th St
NW%

}}
\newcommand{\headerstyled}{%
{}
}
\newcommand{\footerstyled}{%
{\large{Footer}}
}
\newcommand{\datestyled}{%
{\fontsize{14}{16.8}\selectfont
March 14, 2024}
}


\newcommand{\titlepageheaderblock}{\headerstyled}

\newcommand{\titlepagefooterblock}{}

\newcommand{\titlepagedateblock}{
\datestyled
}

%set up blocks so user can specify order
\newcommand{\titleblock}{{

{\titlepagetitleblock}
}

\vspace{4\baselineskip}
}

\newcommand{\authorblock}{{\titlepageauthorblock}

\vspace{2\baselineskip}
}

\newcommand{\affiliationblock}{{\titlepageaffiliationblock}

\vspace{1in}
}

\newcommand{\logoblock}{{\includegraphics[width=0.15\textheight]{img/SREB
white.png}}

\vspace{1\baselineskip}
}

\newcommand{\footerblock}{{\titlepagefooterblock}

\vspace{1pt}
}

\newcommand{\dateblock}{{\titlepagedateblock}

\vspace{0pt}
}

\newcommand{\headerblock}{}
\newgeometry{top=0in,bottom=0in,right=0in,left=0pt}

\thispagestyle{empty} % no page numbers on titlepages


\newcommand{\vrulecode}{\noindent\colorbox{srebblue}{\begin{minipage}[b][.996\textheight][s]{\vrulewidth}

% all aligned .375in
\vspace{2.75in}

\hspace{0.375in}\logoblock

\vspace{5.75in}

% size is 14.4
\hspace{0.375in}{\makebox[\textwidth][l]{\bfseries{\Large{\color{white}Southern}}}}

\hspace{0.375in}{\makebox[\textwidth][l]{\bfseries{\Large{\color{white}Regional}}}}

\hspace{0.375in}{\makebox[\textwidth][l]{\bfseries{\Large{\color{white}Educational}}}}

\hspace{0.375in}{\makebox[\textwidth][l]{\bfseries{\Large{\color{white}Board}}}}

\vspace{0.25in}

\hspace{0.375in}{\makebox[\textwidth][l]{\bfseries{\Large{\color{white}sreb.org}}}}

% 0.625 space below
\vspace{0.625in}

\end{minipage}}}

\newlength{\vrulewidth}
\setlength{\vrulewidth}{2.5in}
\newlength{\B}
\setlength{\B}{\ifdim\vrulewidth > 0pt 0.375in\else 0pt\fi}
\newlength{\minipagewidth}
\ifthenelse{\equal{left}{left} \OR \equal{left}{right} }
{% True case
\setlength{\minipagewidth}{\textwidth - \vrulewidth - \B - 0.1\textwidth}
}{
\setlength{\minipagewidth}{\textwidth - 2\vrulewidth - 2\B - 0.1\textwidth}
}
\ifthenelse{\equal{left}{left} \OR \equal{left}{leftright}}
{% True case
\flushleft % needed for the minipage to work
\vrulecode
\hspace{\B}
}{%
\raggedright % else it is right only and width is not 0
}
% [position of box][box height][inner position]{width}
% [s] means stretch out vertically; assuming there is a vfill
\begin{minipage}[b][\textheight][s]{\minipagewidth}
\vspace{2.75in}
\titlepagepagealign
\titleblock

\authorblock

\dateblock

\affiliationblock

\vfill

\footerblock
\par

\end{minipage}\ifthenelse{\equal{left}{right} \OR \equal{left}{leftright} }{
\hspace{\B}
\vrulecode}{}
\clearpage
\restoregeometry
%%% TITLE PAGE END
\end{titlepage}
\setcounter{page}{1}

%%%%% end titlepage extension code

\section{Data Sources}\label{data-sources}

This report provides an overview of literacy in SREB states using
proprietary data from schools that have received a Curriculum
Instruction Review from SREB in the past 3 years, as well as national
and international literacy assessments.

\begin{itemize}
\tightlist
\item
  SREB Curriculum Instruction Review (CIR) survey response data
  collected from middle grade, high school, and technology center
  students and teachers
\item
  National Assessment of Educational Progress (NAEP)
\item
  Progress in International Reading Literacy Study (PIRLS)
\end{itemize}

\section{SREB Curriculum Instruction Review
(CIR)}\label{sreb-curriculum-instruction-review-cir}

\subsection{Summary}\label{summary}

The Curriculum Instruction Review offers schools a comprehensive view of
school and classroom practices. Data on literacy instruction is
collected in survey responses from both students and teachers. Responses
to a series of questions related to literacy instruction are measured on
a scale of 1 to 4, where 4 would indicate the most powerful literacy
instruction. See the Supplemental Figures Section for examples of
questions on the survey.

This data has been collected across 3 years of SREB surveys and 124
schools, including 21,056 students and 2,683 teachers. While not a
comprehensive look at literacy instruction throughout all SREB states,
it may provide a snapshot of literacy instruction within some SREB
schools.

\subsection{Pooled Response
Distribution}\label{pooled-response-distribution}

\begin{center}
\includegraphics{Literacy_files/figure-pdf/unnamed-chunk-2-1.pdf}
\end{center}

\begin{center}
\includegraphics{Literacy_files/figure-pdf/unnamed-chunk-3-1.pdf}
\end{center}

\subsection{Yearly Response Averages}\label{yearly-response-averages}

Survey data is collected every year, however no schools reported data
for more than one year. In other words, each year is independent from
the other.

\begin{center}
\includegraphics{Literacy_files/figure-pdf/unnamed-chunk-5-1.pdf}
\end{center}

\begin{center}
\includegraphics{Literacy_files/figure-pdf/unnamed-chunk-6-1.pdf}
\end{center}

\newpage{}

\subsection{Survey Scope}\label{survey-scope}

\begin{table}
\centering
\begin{tabular}[t]{lr}
\toprule
 & N\\
\midrule
Alabama & 15\\
Indiana & 2\\
Kentucky & 2\\
Mississippi & 1\\
Missouri & 15\\
New Mexico & 59\\
North Carolina & 1\\
Oklahoma & 2\\
Pennsylvania & 6\\
Texas & 2\\
Vermont & 1\\
Washington & 1\\
West Virginia & 16\\
\bottomrule
\end{tabular}
\end{table}

With a limited number of schools for each state (except for New Mexico),
we can turn to a more complete dataset from the Department of Education
for state-level data.

\newpage{}

\section{National Assessment of Educational Progress
(NAEP)}\label{national-assessment-of-educational-progress-naep}

\subsection{Summary}\label{summary-1}

The National Assessment of Educational Progress (NAEP) provides
important information about student academic achievement and learning
experiences in various subjects. Also known as The Nation's Report Card,
NAEP has provided meaningful results to improve education policy and
practice since 1969. Results are available for the nation, states, and
27 urban districts for 4th, 8th, and 12th grade levels. The NAEP is
administered by the National Center for Education Statistics (NCES),
within the U.S. Department of Education and the Institute of Education
Sciences (U.S. Department of Education).

Using the composite reading score, we report the statewide mean and
percent of students that meet literacy
\href{https://nces.ed.gov/nationsreportcard/reading/achieve.aspx}{at or
above the basic level}.

\begin{center}
\includegraphics{Literacy_files/figure-pdf/unnamed-chunk-9-1.pdf}
\end{center}

\begin{center}
\includegraphics{Literacy_files/figure-pdf/unnamed-chunk-10-1.pdf}
\end{center}

Since 2015, there has been a slight decrease in national literacy
measures across the board.

Due to data unavailability, grades 4 and 8 display current data from
2022, grade 12 shows current data from 2019.

\begin{longtable}[]{@{}lrrrr@{}}
\toprule\noalign{}
NAEP Measure & Grade & 2015 & Current & \% Change \\
\midrule\noalign{}
\endhead
\bottomrule\noalign{}
\endlastfoot
Average Literacy Score & 4 & 222.5 & 217.5 & -2.3\% \\
Average Literacy Score & 8 & 265.4 & 260.5 & -1.9\% \\
Average Literacy Score & 12 & 287.0 & 285.5 & -0.5\% \\
Percent At Basic Level & 4 & 68.8 & 62.6 & -9.1\% \\
Percent At Basic Level & 8 & 76.0 & 69.7 & -8.3\% \\
Percent At Basic Level & 12 & 72.0 & 70.1 & -2.6\% \\
\end{longtable}

\subsection{Percentage of 4th Grade Students At or Above Basic Level
(2022)}\label{percentage-of-4th-grade-students-at-or-above-basic-level-2022}

At the 4th grade level, the majority of SREB states ranked below the
national average of 62\% achieving basic level literacy.

\begin{figure}

\begin{minipage}{0.50\linewidth}
\includegraphics{Literacy_files/figure-pdf/unnamed-chunk-13-1.pdf}\end{minipage}%
%
\begin{minipage}{0.50\linewidth}

\centering
\begin{tabular}[t]{ll}
\toprule
State & Percentage\\
\midrule
Florida & 70.6\%\\
Mississippi & 63.4\%\\
\cellcolor{gray}{\textcolor{white}{\em{National Avg.}}} & \cellcolor{gray}{\textcolor{white}{\em{62.6\%}}}\\
Kentucky & 62.0\%\\
North Carolina & 61.2\%\\
South Carolina & 60.9\%\\
Georgia & 60.8\%\\
Virginia & 59.9\%\\
Tennessee & 59.0\%\\
Alabama & 58.7\%\\
Texas & 58.4\%\\
Arkansas & 58.1\%\\
Louisiana & 57.4\%\\
Maryland & 56.5\%\\
Oklahoma & 55.1\%\\
Delaware & 53.2\%\\
West Virginia & 52.3\%\\
\bottomrule
\end{tabular}

\end{minipage}%

\end{figure}%

\newpage{}

\subsection{Percentage of 8th Grade Students At or Above Basic Level
(2022)}\label{percentage-of-8th-grade-students-at-or-above-basic-level-2022}

At the 8th grade level, all SREB states ranked below the national
average of 70\% achieving basic level literacy. Further, where we would
expect to see an increase in literacy between 4th and 8th grade, some
states like Florida and Mississippi have a lower percentage of literate
8th graders than literate 4th graders.

\begin{figure}

\begin{minipage}{0.50\linewidth}
\includegraphics{Literacy_files/figure-pdf/unnamed-chunk-14-1.pdf}\end{minipage}%
%
\begin{minipage}{0.50\linewidth}

\centering
\begin{tabular}[t]{ll}
\toprule
State & Percentage\\
\midrule
\cellcolor{gray}{\textcolor{white}{\em{National Avg.}}} & \cellcolor{gray}{\textcolor{white}{\em{69.7\%}}}\\
Georgia & 68.8\%\\
Florida & 68.6\%\\
Virginia & 68.5\%\\
Kentucky & 68.2\%\\
Tennessee & 67.2\%\\
Maryland & 66.9\%\\
Louisiana & 66.3\%\\
Texas & 65.8\%\\
North Carolina & 65.7\%\\
Arkansas & 64.2\%\\
Mississippi & 62.9\%\\
South Carolina & 62.7\%\\
Oklahoma & 62.3\%\\
Delaware & 62.2\%\\
Alabama & 60.9\%\\
West Virginia & 60.0\%\\
\bottomrule
\end{tabular}

\end{minipage}%

\end{figure}%

\newpage{}

\subsection{State Trend Table}\label{state-trend-table}

\begin{tabular}{llll}
\toprule
 & \makecell[l]{\hspace{5pt} \makecell[l]{ Percentage of Students \\ At or Above Basic Level} \hspace{45pt} \makecell[l]{\textcolor{domlightblue}{Grade 8} \\ \textcolor{srebblue}{Grade 4}}} & 2005-2022 & 2015-2022\\
\midrule
Alabama & \raisebox{-.4\height}{\includegraphics[width=8cm]{img/fig-alabama.png}} & \makecell[r]{\textcolor{domlightblue}{ -2.8\%}\\\textcolor{srebblue}{ 11.3\%}} & \makecell[r]{\textcolor{domlightblue}{-14.8\%}\\\textcolor{srebblue}{ -9.1\%}}\\
\midrule
Arkansas & \raisebox{-.4\height}{\includegraphics[width=8cm]{img/fig-arkansas.png}} & \makecell[r]{\textcolor{domlightblue}{ -7.0\%}\\\textcolor{srebblue}{ -7.0\%}} & \makecell[r]{\textcolor{domlightblue}{ -8.4\%}\\\textcolor{srebblue}{-10.6\%}}\\
\midrule
Delaware & \raisebox{-.4\height}{\includegraphics[width=8cm]{img/fig-delaware.png}} & \makecell[r]{\textcolor{domlightblue}{-22.4\%}\\\textcolor{srebblue}{-27.5\%}} & \makecell[r]{\textcolor{domlightblue}{-15.3\%}\\\textcolor{srebblue}{-24.1\%}}\\
\midrule
Florida & \raisebox{-.4\height}{\includegraphics[width=8cm]{img/fig-florida.png}} & \makecell[r]{\textcolor{domlightblue}{  4.0\%}\\\textcolor{srebblue}{  9.0\%}} & \makecell[r]{\textcolor{domlightblue}{ -8.5\%}\\\textcolor{srebblue}{ -5.3\%}}\\
\midrule
Georgia & \raisebox{-.4\height}{\includegraphics[width=8cm]{img/fig-georgia.png}} & \makecell[r]{\textcolor{domlightblue}{  2.9\%}\\\textcolor{srebblue}{  4.3\%}} & \makecell[r]{\textcolor{domlightblue}{ -6.4\%}\\\textcolor{srebblue}{-10.7\%}}\\
\midrule
Kentucky & \raisebox{-.4\height}{\includegraphics[width=8cm]{img/fig-kentucky.png}} & \makecell[r]{\textcolor{domlightblue}{ -9.6\%}\\\textcolor{srebblue}{ -4.4\%}} & \makecell[r]{\textcolor{domlightblue}{-12.5\%}\\\textcolor{srebblue}{-16.7\%}}\\
\midrule
Louisiana & \raisebox{-.4\height}{\includegraphics[width=8cm]{img/fig-louisiana.png}} & \makecell[r]{\textcolor{domlightblue}{  3.8\%}\\\textcolor{srebblue}{  8.3\%}} & \makecell[r]{\textcolor{domlightblue}{  0.4\%}\\\textcolor{srebblue}{ -9.0\%}}\\
\midrule
Maryland & \raisebox{-.4\height}{\includegraphics[width=8cm]{img/fig-maryland.png}} & \makecell[r]{\textcolor{domlightblue}{ -3.4\%}\\\textcolor{srebblue}{-12.6\%}} & \makecell[r]{\textcolor{domlightblue}{-12.2\%}\\\textcolor{srebblue}{-16.5\%}}\\
\midrule
Mississippi & \raisebox{-.4\height}{\includegraphics[width=8cm]{img/fig-mississippi.png}} & \makecell[r]{\textcolor{domlightblue}{  4.2\%}\\\textcolor{srebblue}{ 32.5\%}} & \makecell[r]{\textcolor{domlightblue}{  0.2\%}\\\textcolor{srebblue}{  4.9\%}}\\
\midrule
North Carolina & \raisebox{-.4\height}{\includegraphics[width=8cm]{img/fig-north-carolina.png}} & \makecell[r]{\textcolor{domlightblue}{ -4.7\%}\\\textcolor{srebblue}{ -0.4\%}} & \makecell[r]{\textcolor{domlightblue}{ -8.4\%}\\\textcolor{srebblue}{-15.6\%}}\\
\midrule
Oklahoma & \raisebox{-.4\height}{\includegraphics[width=8cm]{img/fig-oklahoma.png}} & \makecell[r]{\textcolor{domlightblue}{-13.8\%}\\\textcolor{srebblue}{ -8.6\%}} & \makecell[r]{\textcolor{domlightblue}{-17.9\%}\\\textcolor{srebblue}{-22.1\%}}\\
\midrule
South Carolina & \raisebox{-.4\height}{\includegraphics[width=8cm]{img/fig-south-carolina.png}} & \makecell[r]{\textcolor{domlightblue}{ -6.4\%}\\\textcolor{srebblue}{  6.1\%}} & \makecell[r]{\textcolor{domlightblue}{-12.1\%}\\\textcolor{srebblue}{ -5.6\%}}\\
\midrule
Tennessee & \raisebox{-.4\height}{\includegraphics[width=8cm]{img/fig-tennessee.png}} & \makecell[r]{\textcolor{domlightblue}{ -4.9\%}\\\textcolor{srebblue}{ -0.4\%}} & \makecell[r]{\textcolor{domlightblue}{-11.2\%}\\\textcolor{srebblue}{-10.2\%}}\\
\midrule
Texas & \raisebox{-.4\height}{\includegraphics[width=8cm]{img/fig-texas.png}} & \makecell[r]{\textcolor{domlightblue}{ -4.8\%}\\\textcolor{srebblue}{ -8.6\%}} & \makecell[r]{\textcolor{domlightblue}{ -9.2\%}\\\textcolor{srebblue}{ -8.8\%}}\\
\midrule
Virginia & \raisebox{-.4\height}{\includegraphics[width=8cm]{img/fig-virginia.png}} & \makecell[r]{\textcolor{domlightblue}{-12.6\%}\\\textcolor{srebblue}{-16.6\%}} & \makecell[r]{\textcolor{domlightblue}{-11.3\%}\\\textcolor{srebblue}{-18.9\%}}\\
\midrule
West Virginia & \raisebox{-.4\height}{\includegraphics[width=8cm]{img/fig-west-virginia.png}} & \makecell[r]{\textcolor{domlightblue}{-10.0\%}\\\textcolor{srebblue}{-13.6\%}} & \makecell[r]{\textcolor{domlightblue}{-17.0\%}\\\textcolor{srebblue}{-18.3\%}}\\
\bottomrule
\end{tabular}

\newpage{}

\section{PIRLS}\label{pirls}

IEA's PIRLS (Progress in International Reading Literacy Study) is an
ongoing international assessment program of students' reading
achievement in their fourth year of schooling---an important transition
point in their development as readers.

Here we provide some select data from the most recent PIRLS 2021
International Report. The full report can be downloaded
\href{https://www.iea.nl/publications/study-reports/international-reports-iea-studies/pirls-2021-international-results}{here}.
The 2021 Report encountered data collection difficulties due to
COVID-19, which is addressed in detail in the report. 78 schools from
the U.S. participated in the assessment.

The U.S. ranked 11th overall in the assessment.

\begin{figure}[H]

{\centering \includegraphics{img/PIRLS1.3.png}

}

\caption{Top 20 Scores, from Exhibit 1.3}

\end{figure}%%
\begin{figure}[H]

{\centering \includegraphics{img/PIRLS2.2.1.png}

}

\caption{Score Trend in the United States, from Exhibit 2.2.1.}

\end{figure}%

18\% of U.S. Students achieved a score above 625 in the Advanced
Benchmark level. The U.S is ranked 7th in this category.

\begin{figure}[H]

{\centering \includegraphics{img/PIRLS4.2.png}

}

\caption{Score Benchmarks, from Exhibit 4.2}

\end{figure}%

\section{Supplemental Figures}\label{supp-figs}

\subsection{SREB CIR Survey Student
Questions}\label{sreb-cir-survey-student-questions}

\includegraphics{data/PowerfulLiteracyInstruction.png}

\subsection{SREB CIR Survey Teacher
Questions}\label{sreb-cir-survey-teacher-questions}

\includegraphics{data/PowerfulLiteracyInstructionTeachers.png}

\section{References}\label{references}

Mullis, I.V.S., von Davier, M., Foy, P., Fishbein, B., Reynolds, K.A.,
\& Wry, E. (2023). PIRLS 2021 International Results in Reading. Boston
College, TIMSS \& PIRLS International Study Center.
https://doi.org/10.6017/lse.tpisc.tr2103.kb5342.

U.S. Department of Education. Institute of Education Sciences, National
Center for Education Statistics, National Assessment of Educational
Progress (NAEP), 2022 Reading Assessment.



\end{document}
